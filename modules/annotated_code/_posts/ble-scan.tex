\documentclass{article}
\usepackage{amsmath}
\usepackage{algorithm2e}
\begin{document}
    
By looking at the figure it is clear that the value at node $A$ is the sum of the values of its children.
The index of node $A$ is at $2^{h+1}-1$ which is the same as its right child. The left child of $A$ has index $2^h-1$. Let $S$ be the array then 
\begin{align*}
    S[2^{h+1}-1]=S[2^h-1]+S[2^{h+1}-1]
\end{align*}
For node $B$ it is the same except we shift the indices by $2^{h+1}$ to the right.
Therefore for nodes at level $h+1$ we have 
\begin{align*}
    S[k+2^{h+1}-1]=S[k+2^h-1]+S[k+2^{h+1}-1]
\end{align*}
where $k$ starts with 0 and is incremented by $2^{h+1}$.
\begin{algorithm*}
    \DontPrintSemicolon
    \ForAll{$k=0,k<n$ multiples of $2^{h+1}$}{
        $S[k+2^{h+1}-1]=S[k+2^h-1]+S[k+2^{h+1}-1]$\;
    }
\end{algorithm*}
The above is performed for all values of $h$.

\begin{algorithm*}
    \DontPrintSemicolon
    \For{$h=0$ \KwTo $\log n -1$}{
    \For{$k=0,k<n$ multiples of $2^{h+1}$}{
        $S[k+2^{h+1}-1]=S[k+2^h-1]+S[k+2^{h+1}-1]$\;
    }
    }
\end{algorithm*}
Now, multiples of $2^{h+1}$ can be write as $\alpha 2^{h+1}$ with $\alpha=0,\ldots,d-1$ with $d=\frac{n}{2^{h+1}}$.
\begin{algorithm*}
    \DontPrintSemicolon
    \For{$d=\frac{n}{2}$ \KwTo $1$ }{
        \For{$\alpha=0$ \KwTo $d-1$}{
            $S[(2\alpha+2)s-1]+=S[(2\alpha +1)s-1]$\;
            $\alpha=\alpha+1$\;
        }
    $s=2*s$\;
    $d=d/2$\;
}
\end{algorithm*}

\end{document}