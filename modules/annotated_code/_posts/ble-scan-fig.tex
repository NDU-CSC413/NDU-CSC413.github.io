% Red-black tree
% Author: Madit
\documentclass{article}
\usepackage{tikz}
%%%<
\usepackage{verbatim}
\usepackage[active,tightpage]{preview}
\usetikzlibrary{calc}
\PreviewEnvironment{tikzpicture}
\setlength{\PreviewBorder}{10pt}%
%%%>
\begin{comment}
:Title: Red-black tree
:Tags: Trees;Graphs
:Author: Madit
:Slug: red-black-tree

A red-black tree is a special type of binary tree, used in computer science
to organize pieces of comparable data, such as text fragments or numbers.
(Wikipedia)
\end{comment}
\usetikzlibrary{arrows}

\tikzset{
  treenode/.style = {align=center, inner sep=0pt, text centered,
    font=\sffamily},
  arn_n/.style = {treenode, circle, white, font=\sffamily\bfseries, draw=black,
    fill=black, text width=1.5em},% arbre rouge noir, noeud noir
  arn_r/.style = {treenode, circle, black, draw=black, 
    text width=1.5em, very thick},% arbre rouge noir, noeud rouge
  arn_x/.style = {treenode, rectangle, draw=black,text width=.5 em,
    minimum width=0.5em, minimum height=0.5em}% arbre rouge noir, nil
}

\begin{document}
\begin{tikzpicture}[<-,>=stealth',level/.style={sibling distance = 5cm/#1,
  level distance = 1.5cm}] 
\node [arn_r] {28}
    child{ node [arn_r] {6} 
            child{ node [arn_r] {1} 
            	child{ node [arn_r] {0}}% edge from parent node[above left]
                         %{$x$}} %for a named pointer
							child{ node [arn_r] {1}}
            }
            child{ node [arn_r] {5}
							child{ node [arn_r] {2}}
							child{ node [arn_r] {3}}
            }                            
    }
    child{ node [arn_r] {22}
            child{ node [arn_r] {9} 
							child{ node [arn_r] {4}}
							child{ node [arn_r] {5}}
            }
            child{ node [arn_r] {13}
							child{ node [arn_r] {6}}
							child{ node [arn_r] {7}}
            }
		}
; 
\end{tikzpicture}
\begin{tikzpicture}[level/.style={sibling distance = 3cm/#1,
  level distance = 1.5cm}] 
 % \draw (0,0) node {upsweep};
 \node [arn_x]{}
child {
      node [arn_x]{\small A}
        child { 
          node (B) [arn_x]{}
            child {node (BA) [arn_x]{}}
            child {node (BB) [arn_x]{}}
        }
        child {
          node (C) [arn_x]{}
            child {node (CA) [arn_x]{}}
            child {node (CB) [arn_x]{}}
        }
      }
child {
  node [arn_x]{B}
    child { 
      node (D) [arn_x]{}
        child {node (DA) [arn_x]{}}
        child {node (DB) [arn_x]{}}
    }
    child {
      node (E) [arn_x]{}
        child {node (EA) [arn_x]{}}
        child {node (EB) [arn_x]{}}
    }
  }
;
\draw (BA) -- (BB);
\draw (CA) -- (CB);
\draw[<->] ($(BA)+(0,-0.3)$) -- ($(BB)+(0,-0.3)$) node[midway,below]{$2^h$};
%\draw[<->] ($(BA)+(-0.3,0)$) -- ($((BA).x-0.3,(B).y)$);
\draw[<->]($(EB)+(0.3,0)$)--($(EB)+(0.3,3)$) node[midway,right]{$h+1$};
\draw[<->]($(BA)+(-0.3,0)$)--($(BA)+(-0.3,1.5)$) node[midway,left]{$h$};

\end{tikzpicture}

We check
\end{document}